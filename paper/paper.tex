% THIS IS SIGPROC-SP.TEX - VERSION 3.1
% WORKS WITH V3.2SP OF ACM_PROC_ARTICLE-SP.CLS
% APRIL 2009
%
% It is an example file showing how to use the 'acm_proc_article-sp.cls' V3.2SP
% LaTeX2e document class file for Conference Proceedings submissions.
% ----------------------------------------------------------------------------------------------------------------
% This .tex file (and associated .cls V3.2SP) *DOES NOT* produce:
%       1) The Permission Statement
%       2) The Conference (location) Info information
%       3) The Copyright Line with ACM data
%       4) Page numbering
% ---------------------------------------------------------------------------------------------------------------
% It is an example which *does* use the .bib file (from which the .bbl file
% is produced).
% REMEMBER HOWEVER: After having produced the .bbl file,
% and prior to final submission,
% you need to 'insert'  your .bbl file into your source .tex file so as to provide
% ONE 'self-contained' source file.
%
% Questions regarding SIGS should be sent to
% Adrienne Griscti ---> griscti@acm.org
%
% Questions/suggestions regarding the guidelines, .tex and .cls files, etc. to
% Gerald Murray ---> murray@hq.acm.org
%
% For tracking purposes - this is V3.1SP - APRIL 2009

\documentclass{acm_proc_article-sp}

\newdef{definition}{Definition}

\begin{document}

\title{EasyMerge - A New Tool for Code Clones Refactoring}
%
% You need the command \numberofauthors to handle the 'placement
% and alignment' of the authors beneath the title.
%
% For aesthetic reasons, we recommend 'three authors at a time'
% i.e. three 'name/affiliation blocks' be placed beneath the title.
%
% NOTE: You are NOT restricted in how many 'rows' of
% "name/affiliations" may appear. We just ask that you restrict
% the number of 'columns' to three.
%
% Because of the available 'opening page real-estate'
% we ask you to refrain from putting more than six authors
% (two rows with three columns) beneath the article title.
% More than six makes the first-page appear very cluttered indeed.
%
% Use the \alignauthor commands to handle the names
% and affiliations for an 'aesthetic maximum' of six authors.
% Add names, affiliations, addresses for
% the seventh etc. author(s) as the argument for the
% \additionalauthors command.
% These 'additional authors' will be output/set for you
% without further effort on your part as the last section in
% the body of your article BEFORE References or any Appendices.

\numberofauthors{3} %  in this sample file, there are a *total*
% of EIGHT authors. SIX appear on the 'first-page' (for formatting
% reasons) and the remaining two appear in the \additionalauthors section.
%
\author{
% You can go ahead and credit any number of authors here,
% e.g. one 'row of three' or two rows (consisting of one row of three
% and a second row of one, two or three).
%
% The command \alignauthor (no curly braces needed) should
% precede each author name, affiliation/snail-mail address and
% e-mail address. Additionally, tag each line of
% affiliation/address with \affaddr, and tag the
% e-mail address with \email.
%
% 1st. author
\alignauthor
Shengying Pan\\
       \affaddr{School of Computer Science}\\
       \affaddr{University of Waterloo}\\
       \email{s5pan@uwaterloo.ca}
% 2nd. author
\alignauthor
Haocheng Qin\\
       \affaddr{School of Computer Science}\\
       \affaddr{University of Waterloo}\\
       \email{h7qin@uwaterloo.ca}
% 3rd. author
\alignauthor Yahui Chen\\
       \affaddr{School of Computer Science}\\
       \affaddr{University of Waterloo}\\
       \email{y556chen@uwaterloo.ca}
\and  % use '\and' if you need 'another row' of author names
}
% There's nothing stopping you putting the seventh, eighth, etc.
% author on the opening page (as the 'third row') but we ask,
% for aesthetic reasons that you place these 'additional authors'
% in the \additional authors block, viz.
\additionalauthors{Additional authors: John Smith (The Th{\o}rv{\"a}ld Group,
email: {\texttt{jsmith@affiliation.org}}) and Julius P.~Kumquat
(The Kumquat Consortium, email: {\texttt{jpkumquat@consortium.net}}).}
\date{30 July 1999}
% Just remember to make sure that the TOTAL number of authors
% is the number that will appear on the first page PLUS the
% number that will appear in the \additionalauthors section.

\maketitle
\begin{abstract}
Code clones are common in medium to large scale software projects. Oftentimes, unnecessary clones cause troubles to code base maintenance and code reusability. 
Over past decades, many techniques and approaches have been proposed to detect code clones. However, how to refactor clones is still a very challenging topic to software
engineers. Even text-wise identical code clones can be semantically different when they are referring variables and calling functions outside. And the problem is more 
complex when scopes and dependencies are considered. Furthermore, not all clones can be refactored as they may be part of tests or needed to maintain dependencies across multiple
libraries. As a result, we need adaptive clone refactoring tools that can locate unnecessary clones and alert possible implications in the procedure to help
software engineers be more efficient and make less mistakes in the refactoring process.

In this paper, we introduce EasyMerge, a new tool to refactor code clones. EasyMerge is built on top of AST-based clone detection algorithm
\end{abstract}

% A category with the (minimum) three required fields
%\category{H.4}{Information Systems Applications}{Miscellaneous}
%A category including the fourth, optional field follows...
\category{D.2.8}{Software Engineering}{Metrics}[complexity measures, performance measures]

\terms{Software Engineering, Recommendation System}

\keywords{Software Engineering, Clone Detection, Code Refactoring, Recommendation System, Python} % NOT required for Proceedings

\section{Introduction}
In software development, it's very common seeing developers reuse code fragments by copying and pasting with or without minor adaptation.
Moreover, for large scale projects, developers are often too lazy to browse existing source files so that they may rewrite similar or even identical functions which
were already in the code base. As a result, software systems often contain sections of code that are very similar, called code clones.

Previous research shows that a significant fraction (between 7\% and 23\%) of the code in a typical software system has been cloned \cite{baker} \cite{roy1}. Many code clones
in code bases are unnecessary duplications. 
Code duplication can be a significant drawback, leading to bad design, and increased probability of bug occurrence and propagation. As a result, it can significantly
increase maintenance cost, and form a barrier for software evolution. By detecting, categorizing,
and removing code clones, we can produce easier to understand, cleaner, and more reusable code.

Clone detection has been an avid research topic in the field of software engineering for decades. Fortunately, several automated techniques for detecting code clones
have already been proposed. However, how to deal with detected clones, e.g. how to distinguish necessary clones from unnecessary ones and how to refactor code to remove
unnecessary clones still remain a big problem in not only commercial but also academic domain. As a result, in this paper, we try to classify code clones and build
a recommendation system called EasyMerge to help developers merge unnecessary clones on top of current state-of-the-art clone detection approach.

More specifically, we pick CloneDigger \cite{bulychev}, an anti-unification duplicate code detection tool as our underlying clone detection approach. CloneDigger is one of the best available 
clone detection tools currently for its overall performance, coverage of multiple clone types, and availability. 
EasyMerge integrates CloneDigger as the pre-processing tool, analyze its output clone pairs, and recommend possible merges which can remove unnecessary clones without changing functionality of code base, creating reference conflicts, nor causing troubles to future code understanding and development.

The rest of the paper is structured as follows: we first go through the basics, background, and current state of clone detection and code clone refactoring in general.
Then we introduce and discuss the fundamentals of CloneDigger and the anti-unification algorithm it is using to detect clones. Afterwards, we explain EasyMerge's
work flow and underlying techniques. And at the end, we set up testing environment and discuss the experimental results of running EasyMerge against several
open source projects of different scales.

\section{Background}
Roy, Cordy, and Koschke have done a great work \cite{roy2} writing an overview paper explaining the basics of clone detection, and providing a complete comparison of essential
strengths and weaknesses of both individual tools and techniques and alternative approaches in general. It gives us all the needed preliminaries to focus on clone refactoring rather
than spending time working on the detection part. We begin with a basic introduction to clone detection terminology in Roy's paper.

\begin{definition}
(Code Fragment). A code fragment (CF) is any sequence of code lines (with or without comments). It can be of any granularity, e.g., function
definition, begin-end block, or sequence of statements. A $CF$ is identified by its file name and begin-end line numbers in the original code base
and is denoted as a triple ($CF.FileName$, $CF.BeginLine$, $CF.EndLine$).
\end{definition}

\begin{definition}
(Code Clone). A code fragment $CF2$ is a clone of another code fragment $CF1$ if they are similar by some given definition of similarity, that is, 
$f(CF1) = f(CF2)$ where $f$ is the similarity function (see clone types below). Two fragments that are similar to each other form a clone pair
$(CF1, CF2)$, and when many fragments are similar, they form a clone class or clone group.
\end{definition}

\begin{definition}
(Clone Types). There are two main kinds of similarity between code fragments. Fragments can be similar based on the similarity of their program text,
or they can be similar based on their functionality (independent of their text). The first kind of clone is often the result of copying a code fragment and
pasting into another location. In the following we provide the types of clones based on both the textual (Type 1 to 3)\cite{bellon} and functional (Type 4)\cite{gabel, komondoor} similarities:

\begin{itemize}
\item {\bf Type-1:} Identical code fragments except for variations in whitespace, layout and comments.
\item {\bf Type-2:} Syntactically identical fragments except for variations in identifiers, literals, types, whitespace, layout and comments.
\item {\bf Type-3:} Copied fragments with further modifications such as changed, added or removed statements, in addition to variations in identifiers, literals, types, whitespace, layout and comments.
\item {\bf Type-4:} Two or more code fragments that perform the same computation but are implemented by different syntactic variants.
\end{itemize}
\end{definition}

\section{Clone Code Detection}

\section{EasyMerge}

\section{Experimental Results}

\section{Conclusions}

\section{Future Work}

%ACKNOWLEDGMENTS are optional
\section{Acknowledgments}
This section is optional; it is a location for you
to acknowledge grants, funding, editing assistance and
what have you.  In the present case, for example, the
authors would like to thank Gerald Murray of ACM for
his help in codifying this \textit{Author's Guide}
and the \textbf{.cls} and \textbf{.tex} files that it describes.

%
% The following two commands are all you need in the
% initial runs of your .tex file to
% produce the bibliography for the citations in your paper.
\bibliographystyle{abbrv}
\bibliography{paper}  % sigproc.bib is the name of the Bibliography in this case
% You must have a proper ".bib" file
%  and remember to run:
% latex bibtex latex latex
% to resolve all references
%
% ACM needs 'a single self-contained file'!
%
%APPENDICES are optional
%\balancecolumns
%\appendix
%Appendix A
%\subsection{References}
%Generated by bibtex from your ~.bib file.  Run latex,
%then bibtex, then latex twice (to resolve references)
%to create the ~.bbl file.  Insert that ~.bbl file into
%the .tex source file and comment out
%the command \texttt{{\char'134}thebibliography}.
%\balancecolumns
% That's all folks!
\end{document}
